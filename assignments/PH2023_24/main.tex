%\iffalse
\let\negmedspace\undefined
\let\negthickspace\undefined
\documentclass[journal,12pt,onecolumn]{IEEEtran}
\usepackage{cite}
\usepackage{amsmath,amssymb,amsfonts,amsthm}
\usepackage{algorithmic}
\usepackage{graphicx}
\usepackage{textcomp}
\usepackage{xcolor}
\usepackage{txfonts}
\usepackage{listings}
\usepackage{enumitem}
\usepackage{mathtools}
\usepackage{gensymb}
\usepackage{comment}
\usepackage[breaklinks=true]{hyperref}
\usepackage{tkz-euclide} 
\usepackage{listings}
\usepackage{gvv}    
\usepackage{enumitem}
\usepackage{amsmath}
\def\inputGnumericTable{}                                 
\usepackage[latin1]{inputenc}                                
\usepackage{color}              \usepackage{circuitikz}                              
\usepackage{array}                                            
\usepackage{longtable}                                       
\usepackage{calc}                                             
\usepackage{multirow}                                         
\usepackage{hhline}                                           
\usepackage{ifthen}                                           
\usepackage{lscape}
\usepackage{tabularx}
\usetikzlibrary{shapes.gates.logic.US, circuits.logic.US}

\newtheorem{theorem}{Theorem}[section]
\newtheorem{problem}{Problem}
\newtheorem{proposition}{Proposition}[section]
\newtheorem{lemma}{Lemma}[section]
\newtheorem{corollary}[theorem]{Corollary}
\newtheorem{example}{Example}[section]
\newtheorem{definition}[problem]{Definition}
\newcommand{\BEQA}{\begin{eqnarray}}
\newcommand{\EEQA}{\end{eqnarray}}
\newcommand{\define}{\stackrel{\triangle}{=}}
\theoremstyle{remark}
\newtheorem{rem}{Remark}
\begin{document}
\bibliographystyle{IEEEtran}
\vspace{3cm}

\title{GATE:PH-24-2023}
\author{EE23BTECH11048 - P Venkata Chanakya $^{}$% <-this % stops a space
}
\maketitle
\bigskip
\textbf{Question}
Which one of the following options is CORRECT for the given logic circuit?
\begin{center}
    \begin{circuitikz}

        % AND gate
        \draw (-2,0) node[and port, anchor=in 1] (and1) {};
        \draw (and1.in 1) -- ++(-0.7,0) node[above] {$P$}node[fill,circle,inner sep=2pt] {} ;
        \draw (and1.in 2) -- ++(-0.7,0) node[below] {$Q$}node[fill,circle,inner sep=2pt] {} ;
        \draw (and1.out) -- ++(2,0);

        % NAND gate
        \draw (0,2) node[nand port, anchor=in 1] (nand1) {};
        \draw (nand1.in 1) -- ++(-0.5,0) -- ++ (0,-0.6) ;
        \draw (nand1.in 2) -- ++(-0.5,0) ;
        \draw (nand1.out) -- ++(0.4,0) -- ++ (0,-1.41);

        % OR gate
        \draw (2,0.27) node[or port, anchor=in 1] (or1) {};
        \draw (or1.in 1) -- ++(-0.1,0);
        \draw (or1.in 2) -- ++(-0.5,0);
        \draw (or1.out) -- ++(0.5,0) node[right] {Y}node[fill,circle,inner sep=2pt] {} ;
        \draw (-2,0) -- ++ (0,1.75) -- ++ (1.43,0);
    \end{circuitikz}
\end{center}
\begin{enumerate}
    \item  $P = 1$, $Q = 1$; $X = 0$
    \item  $P = 1$, $Q = 0$; $X = 1$
    \item  $P = 0$, $Q = 1$; $X = 0$
    \item  $P = 0$, $Q = 0$; $X = 1$
\end{enumerate}
\hfill{PH-2023}\\
\solution\\
$X = \overline{P} + P \cdot Q$

\begin{center}
    
\begin{tabular}{|cc|c|}
\hline
  $P$ & $Q$ & $X$ \\ \hline
  0 & 0 & 1 \\ 
  0 & 1 & 1 \\ 
  1 & 0 & 0 \\ 
  1 & 1 & 1 \\ \hline
\end{tabular}
\end{center}

$\implies 4$ is correct option
\end{document}

