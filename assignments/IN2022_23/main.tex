%\iffalse
\let\negmedspace\undefined
\let\negthickspace\undefined
\documentclass[journal,12pt,onecolumn]{IEEEtran}
\usepackage{cite}
\usepackage{amsmath,amssymb,amsfonts,amsthm}
\usepackage{algorithmic}
\usepackage{graphicx}
\usepackage{textcomp}
\usepackage{xcolor}
\usepackage{txfonts}
\usepackage{listings}
\usepackage{enumitem}
\usepackage{mathtools}
\usepackage{gensymb}
\usepackage{comment}
\usepackage[breaklinks=true]{hyperref}
\usepackage{tkz-euclide} 
\usepackage{listings}
\usepackage{gvv}    
\usepackage{enumitem}
\usepackage{amsmath}
\def\inputGnumericTable{}                                 
\usepackage[latin1]{inputenc}                                
\usepackage{color}              \usepackage{circuitikz}                              
\usepackage{array}                                            
\usepackage{longtable}                                       
\usepackage{calc}                                             
\usepackage{multirow}                                         
\usepackage{hhline}                                           
\usepackage{ifthen}                                           
\usepackage{lscape}
\usepackage{tabularx}
\usetikzlibrary{shapes.gates.logic.US, circuits.logic.US}

\newtheorem{theorem}{Theorem}[section]
\newtheorem{problem}{Problem}
\newtheorem{proposition}{Proposition}[section]
\newtheorem{lemma}{Lemma}[section]
\newtheorem{corollary}[theorem]{Corollary}
\newtheorem{example}{Example}[section]
\newtheorem{definition}[problem]{Definition}
\newcommand{\BEQA}{\begin{eqnarray}}
\newcommand{\EEQA}{\end{eqnarray}}
\newcommand{\define}{\stackrel{\triangle}{=}}
\theoremstyle{remark}
\newtheorem{rem}{Remark}
\begin{document}
\bibliographystyle{IEEEtran}
\vspace{3cm}

\title{GATE:IN-23-2022}
\author{EE23BTECH11048 - P Venkata Chanakya $^{}$% <-this % stops a space
}
\maketitle
\bigskip



\textbf{Question}
The output F of the digital circuit shown can be written in the form $\rule{1cm}{0.15mm}$\\
\begin{center}
\begin{circuitikz}[scale=1]

% First 2-1 MUX
\draw (0,0) node[draw, minimum width=3cm, minimum height=4cm] (mux1) {\parbox{1.6cm}{$2\times1$ \\ MUX}};
\draw (-1.5,1.4)node[right] {$I_0$} -- ++(-1.5,0) node[above] {$1$} ;
\draw (-1.5,-1.4)node[right] {$I_1$} -- ++(-1.5,0) node[above] {$0$};
\draw (0,-2)node[above] {$S_0$} -- ++(0,-1.5) node[left] {$B$};
\draw (mux1.east) -- ++(1.5,0) node[right] {};

% Second 2-1 MUX positioned to the right of the first one
\draw (6,0) node[draw, minimum width=3cm, minimum height=4cm] (mux2) {\parbox{1.6cm}{$2\times1$ \\ MUX}};
\draw (4.5,1.4)node[right] {$I_0$} -- ++(-1.5,0) node[above] {$1$} ;
\draw (4.5,-1.4)node[right] {$I_1$} -- ++(-1.5,0) node[left] {};
\draw (6,-2)node[above] {$S_0$} -- ++(0,-1.5) node[left] {$A$};
\draw (mux2.east) -- ++(1,0) node[right] {$F$};
\draw (3,0) -- ++(0,-1.4);
\end{circuitikz}
\end{center}
\begin{enumerate}
    \item $\overline{A\cdot B}$
    \item $\overline{A}+\overline{B}$
    \item $\overline{A+B}$
    \item $\overline{A} \cdot \overline{B}$
\end{enumerate} 
\hfill{IN 2022}\\
\solution\\
Truth table for the following digital circuit\\
\begin{center}
    
\begin{tabular}{|cc|c|}
\hline
  $P$ & $Q$ & $X$ \\ \hline
  0 & 0 & 1 \\ 
  0 & 1 & 1 \\ 
  1 & 0 & 0 \\ 
  1 & 1 & 1 \\ \hline
\end{tabular}
\end{center}

$\implies A \text{ and } B$ are correct choices
\end{document}

